\documentclass[11pt]{article}

\usepackage[utf8]{inputenc}   
\usepackage{times}            
\usepackage{graphicx}         
\usepackage{amsmath, amssymb} 
\usepackage{hyperref}         
\usepackage{geometry}         
\geometry{margin=1in}

\title{BLIS: Blackbox Inference Simulator}
\author{AI Platform Optimization Team}
\date{\today}

\begin{document}

\maketitle

\begin{abstract}
This paper introduces BLIS, a blackbox inference simulator designed to model request flows and latency in large language model (LLM) inference workloads.
\end{abstract}

\section{Introduction}
\label{sec:introduction}

\section{Related Work}
\label{sec:related-work}

\section{Design overview}
\label{sec:design-overview}

Our simulator is primarily geared towards vLLM. We will describe extensions for other
inference platforms like sglang in Section \ref{sec:future-work}.

\subsection{vLLM overview}
\label{subsec:vllm-overview}
We will focus on handling of an inference request by vLLM. Conceptually, vLLM
has two main components: the API server and the engine core; these operate as separate threads and
communicate through internal queues.

\paragraph{API server thread.}  
The API server is implemented using FastAPI. Its role is to handle client-facing tasks without blocking on actual model execution. 
The API thread is responsible for:
\begin{enumerate}
    \item Receiving and tokenizing incoming requests from clients.
    \item Enqueuing tokenized requests into the scheduler’s pending queue.
    \item Streaming back responses as they become available from the engine.
    \item Detokenizing completed sequences and packaging them into response
          objects.
\end{enumerate}

\paragraph{Engine core thread.}  
The engine core runs the central \emph{inference loop}. This loop is driven by a
scheduler that repeatedly:
\begin{enumerate}
    \item Collects pending requests from the queue, forming a dynamic batch.
    \item Determines whether each request is in a \emph{prefill} (first token)
          or \emph{decode} (subsequent tokens) stage.
    \item Executes the appropriate forward pass on the GPU for that batch.
    \item Updates KV cache blocks, manages allocation and eviction, and applies
          prefix caching optimizations when possible.
    \item Returns generated tokens back to the API layer, either for streaming
          or final response.
\end{enumerate}
We will refer to this engine loop also as the \emph{busy loop} since it continuously
steps through micro-batches at a fine-grained timescale (e.g., every few
microseconds).

\subsection{Simulator Component design}
\label{subsec:simulator-component-design}

\section{Latency model}
\label{sec:latency-model}

\subsection{End-to-End request latency model}
\label{subsec:end-to-end-latency-model}

\subsubsection{Request life-cycle}
\label{subsubsec:request-life-cycle}

Request is emitted by client. 
Request hits the server endpoint.
Request is transferred to the engine wait queue.
Request is scheduled. At this point, it is part of the running batch.
Request is prefilled.
Request is decoded.
Request is detokenized.
Client receives the response.

We need notation for stuff such as ... 
$e_i$: end-to-end latency as seen by the client for request 'i'
$n_i$: network latency from the time the request was emitted by the client, to the time the request was received by the server.

For e.g., $t_{i, in}$ is the number of input tokens. And this affects the tokenization latency.



\subsubsection{Busy-loop latency model}
\label{subsubsec:busy-loop-latency-model}


\section{Estimation}
\label{sec:estimation}

\subsection{Base Estimation with Linear Regression}
\label{subsec:base-estimation-linear-regression}

\subsection{Refinements using Blackbox Optimization}
\label{subsec:refinements-blackbox-optimization}

\section{Evaluation}
\label{sec:evaluation}

\section{Implementation}
\label{sec:implementation}

\section{Limitations of our Approach}
\label{sec:limitations}

Our current approach has several important limitations:

\begin{enumerate}
    \item \textbf{Hardware and model specificity.}  
    The simulator is calibrated for a fixed combination of GPU type and model
    architecture. We do not attempt to generalize across different hardware or
    models. If either changes, the simulator must be retrained with new data.

    \item \textbf{No modeling of pre-emption.}  
    We do not simulate pre-emption of running requests. In practice, triggering
    pre-emption during request handling is rare and avoided by the scheduler
    logic, but its absence in our model means we cannot accurately predict
    latency under pre-emptive scenarios.

    \item \textbf{No modeling of speculative decoding.}  
    We currently omit speculative decoding mechanisms (such as those recently
    introduced in vLLM as an experimental feature).
\end{enumerate}

\section{Future Work}
\label{sec:future-work}

\section{Conclusion}
\label{sec:conclusion}

\bibliographystyle{plain}
\bibliography{blis}

\end{document}
